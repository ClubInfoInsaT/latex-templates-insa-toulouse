\newcommand\tab[1][0.6cm]{\hspace*{#1}} %Create and define tab

\definecolor{lightgray}{gray}{0.85}
\definecolor{lightgrey}{gray}{0.85}
\definecolor{vlg}{gray}{0.85}


%Patch pour utiliser des équations dans les titres sans que hypperref nous insulte.
% Définition cyclique, compile pas. Mais c'est l"idée
%\renewcommand{\chapter}[1]{\chapter{\texorpdfstring{#1}}}
%\renewcommand{\section}[1]{\section{\texorpdfstring{#1}}}
%\renewcommand{\subsection}[1]{\subsection{\texorpdfstring{#1}}}
%\renewcommand{\subsubsection}[1]{\subsubsection{\texorpdfstring{#1}}}

%Chapter No Numbering but appears in TOC
\newcommand{\chapternn}[1]{\phantomsection{}\chapter*{#1}\addcontentsline{toc}{chapter}{#1}}
\newcommand{\sectionnn}[1]{\phantomsection{}\section*{#1}\addcontentsline{toc}{section}{#1}\setcounter{subsection}{0}}
\newcommand{\subsectionnn}[1]{\phantomsection{}\subsection*{#1}\addcontentsline{toc}{subsection}{#1}\setcounter{subsubsection}{0}}
\newcommand{\subsubsectionnn}[1]{\phantomsection{}\subsubsection*{#1}\addcontentsline{toc}{subsubsection}{#1}}

\newcolumntype{L}[1]{>{\raggedright\arraybackslash\hspace{0pt}}p{#1}}
\newcolumntype{R}[1]{>{\raggedleft\arraybackslash\hspace{0pt}}p{#1}}
\newcolumntype{C}[1]{>{\centering\arraybackslash\hspace{0pt}}p{#1}}


\renewcommand\thesection{\arabic{section}}
\renewcommand\thesubsection{\thesection.\arabic{subsection}}

%------- Do not append new commands after :

\hypersetup{
    bookmarksnumbered,      % active les signets (visibles dans le panneau latéral de firefox notamment)
    bookmarksopen=true,     % ouvre directement les sous-parties...
    bookmarksopenlevel=2,   % ... jusqu'au niveau 2
    bookmarksdepth=3,       % affiche seulement jusqu'au niveau 3

    colorlinks=false, % colorise les liens
    linkbordercolor={1 1 1},
    breaklinks=true, % permet le retour à la ligne dans les liens trop longs
    urlcolor=blue, % couleur des hyperliens 
    linkcolor=black,	% couleur des liens internes 
    citecolor=black,	% couleur des références 
    pdftitle={}, % informations apparaissant dans 
    pdfauthor={}, % les informations du document
    pdfsubject={}	% sous Acrobat. 
}