\documentclass[11pt]{article}
\usepackage{graphicx, pdfpages, tikz, hyperref, fancyhdr,  geometry, titlesec, xcolor, csquotes, tocloft, minitoc, helvet}
\usepackage[T1]{fontenc}
\usepackage[french]{babel}

% marges du document
\geometry{
    left=3cm,
    right=3cm,
    top=2cm,
    bottom=3.5cm
}

\pagestyle{fancy}
\fancyhf{} % retirer config de page par défaut
\renewcommand{\headrulewidth}{0pt} % Supprimer la ligne d'en-tête

\fancyfoot[R]{\thepage} % numéro de page à droite
\fancyfoot[L]{{\itshape \titre}} % numéro de page à droite

\newcommand\BackgroundPic{
    \put(0,0){
        \includegraphics[width=\paperwidth,height=\paperheight]{template/assets/template_page.pdf}
    }
}

% redéfinie les sections
\titleformat{\section}
  {\sffamily\LARGE\bfseries\MakeUppercase} % Arial like, grand, gras et majuscule
  {\thesection}{1em}{}

% redéfinie les sous-sections
\titleformat{\subsection}
  {\Large\bfseries} % Définit la taille à \large et en gras
  {\thesubsection}{1em}{}

\renewcommand{\contentsname}{{\sffamily\LARGE\bfseries\MakeUppercase TABLE OF CONTENTS}}
\addto\captionsfrench{\renewcommand{\contentsname}{\sffamily\LARGE\bfseries\MakeUppercase SOMMAIRE}}


\newcommand{\psection}[1]{\phantomsection\section*{#1}\addcontentsline{toc}{section}{#1}}

\newcommand{\aremplir}{{\LARGE \bfseries \textcolor{red}{A REMPLIR}}}

\addto\captionsfrench{
    \renewcommand{\listfigurename}{Liste des figures}%
    \renewcommand{\listtablename}{Liste des tableaux}%
}

\newenvironment{custombox}[1]{% environnement qui permet le retour à la ligne quand ça déborde
    \begin{minipage}{#1}
}{\end{minipage}} % ne pas toucher

\renewcommand{\familydefault}{\sfdefault} % Si vous voulez passer en Arial le texte

\newcommand{\titre}{Nom du document}
\newcommand{\imagecouverture}{example-image}
\newcommand{\firstcouverture}{
    \parbox{\textwidth}{
        \sffamily % arial
        \textbf{Prénom NOM}\\
        Elève Ingénieur de l'INSA Toulouse\\
        Département XX\\
        Spécialité TLS-SEC\\
        Promotion XX\\
        20XX-20XX
    }
}
\newcommand{\secondcouverture}{
    \parbox{\textwidth}{
        \begin{custombox}{9cm}
            \sffamily % arial
            \textbf{INTITULE ICI - EXEMPLE : CONTRIBUTION A LA CONCEPTION A BAS COUT D’ANTENNES 3D}
            \vspace{1em}\\
            \textbf{Lieu du Projet de Fin d'Études ou stage}\\
            Nom de l’entreprise\\
            Adresse de l’entreprise
            \vspace{0.6em}\\
            \textbf{Tuteur du Projet (ou PFE)...}\\
            Prénom NOM du Tuteur du Projet de Fin d'Étude
            \vspace{0.6em}\\
            \textbf{Correspondant pédagogique INSA}\\
            Prénom NOM du Correspondant pédagogique INSA
            \vspace{0.6em}\\
            \textbf{PFE/Stage/Projet soutenu le 00/00/20XX}
        \end{custombox}
    }
}

% bibliographie
\usepackage{biblatex}
\addbibresource{contents/bibliography.bib}


\begin{document}
    \dosecttoc{}
\pagenumbering{Roman} % Numérotation en chiffres romains (i, ii, iii, ...)
\setcounter{page}{1}

\definecolor{couleurcarre}{HTML}{F3F0EC} 

\thispagestyle{empty} % pas de numéro de page sur cette page
\begin{tikzpicture}[remember picture, overlay]
    \node[anchor=south west, inner sep=0] at (current page.south west) {
        \includegraphics[width=\paperwidth,height=\paperheight]{template/assets/template_first_page.pdf}
    };
    
    \fill[fill=couleurcarre]([xshift=-4.7cm, yshift=-4.5cm]current page.north east) rectangle ++(3.5cm, 3.5cm);
    
    % Ajouter du texte à une position spécifique
    \node at (2.1, -4) {\LARGE \bfseries \MakeUppercase{\sffamily \titre}};
    \node at (6.8, -7) {\large \firstcouverture};
    \node at (6.8, -17) {\secondcouverture};
    %\node at (11.5, -12.2) {\includegraphics[height=3cm]{\imagecouverture}};
    \node at (15.05, -0.3) {\includegraphics[width=3cm]{\imagecouverture}};
\end{tikzpicture}
\newpage
\AddToShipoutPicture{\BackgroundPic}  % ne pas toucher
    \begin{tikzpicture}[remember picture, overlay]
  
    \fill[fill=couleurcarre]([xshift=-4.7cm, yshift=-4.5cm]current page.north east) rectangle ++(3.5cm, 3.5cm);
    
    % Ajouter du texte à une position spécifique
    \node at (2.1, -4) {\LARGE \bfseries \MakeUppercase{\sffamily \titre}};
    \node at (6.8, -7) {\large \firstcouverture};
    \node at (6.8, -17) {{\sffamily\secondcouverture}};
    %\node at (11.5, -12.2) {\includegraphics[height=3cm]{\imagecouverture}};
    \node at (15.05, -0.3) {\includegraphics[width=3cm]{\imagecouverture}};
\end{tikzpicture}
\newpage
    
    % commentez les sections qui ne vous concernent pas
    \psection{Note de confidentialité}
Le présent rapport est classé confidentiel. En conséquence, la divulgation de son contenu à une personne extérieure au corps professoral de l’INSA ou à une personne extérieure à l’entreprise \aremplir{} est interdite.
\newpage
    \psection{Remerciements}
Pour leur aide dans la construction de ce travail, je tiens à remercier plusieurs personnes.\\
Qu’elles trouvent ici l’expression de mes plus sincères remerciements pour leurs précieux conseils.\\\\
Pour cela, je tiens tout d’abord à exprimer ma reconnaissance envers\\
Je remercie tout particulièrement\\
Je remercie aussi spécialement\\

\newpage
    Lorem ipsum dolor sit amet, consectetur adipiscing elit. In at rutrum ipsum. Vestibulum tincidunt ipsum tincidunt volutpat convallis. Proin varius, elit id bibendum tempus, tellus nunc blandit nisi, quis hendrerit leo nisi ac dolor. In hendrerit interdum elit, sed faucibus augue eleifend vel. Nullam erat augue, sodales vulputate massa at, ornare hendrerit magna. In vehicula congue mollis. Sed faucibus risus quis metus molestie commodo. Donec malesuada non nunc a gravida. In faucibus sapien sed turpis hendrerit, eu ornare dolor rutrum. Praesent et consectetur sapien, ut interdum ex. Curabitur ornare erat consequat interdum aliquam. Integer pharetra tortor nec facilisis congue. Nullam quis tellus bibendum, posuere magna ut, malesuada arcu. Sed tincidunt, massa id tempus auctor, mi nunc mollis massa, eget fermentum orci mi ac diam.

Vivamus tempus ligula urna. Mauris imperdiet ullamcorper consectetur. Vestibulum pretium neque at vulputate facilisis. Quisque luctus lobortis arcu, ut consequat augue porttitor in. In iaculis odio. 
    
    % Table des matières
\tableofcontents
\thispagestyle{empty} % pas de numéro de page sur cette page
\newpage

\pagenumbering{arabic} % Numérotation en chiffres romains (i, ii, iii, ...)
\setcounter{page}{1} % ne pas toucher
    
    % début du contenu
    \psection{Introduction}
Une introduction

\newpage
\section{Section de contenu}
\subsection{sous section}
Ici je cite une grande référence \cite{test}

\section{Une autre section}
\begin{figure}[h]
    \centering
    \includegraphics[width=0.5\textwidth]{example-image} % Remplacez par votre image
    \caption{Ceci est un exemple de figure.}
    \label{fig:example}
\end{figure}

\begin{table}[h]
    \centering
    \caption{Ceci est un exemple de tableau.}
    \begin{tabular}{|c|c|c|}
        \hline
        Colonne 1 & Colonne 2 & Colonne 3 \\ \hline
        Donnée 1  & Donnée 2  & Donnée 3  \\ \hline
        Donnée 4  & Donnée 5  & Donnée 6  \\ \hline
    \end{tabular}
    \label{tab:example}
\end{table}

\section{Another section}
\subsection{Une sous section}
\subsubsection{Une sous sous section}
Un mot compliqué\footnote{Une note de bas de page}

\newpage
\psection{Conclusion}
Une conclusion
    
    % commentez les sections qui ne vous concernent pas
    \newpage
\psection{Bibliographie}
\printbibliography[heading=none]
    \input{contents/lexique}
    \newpage
\psection{Liste des figures et tableaux}
\listoffigures
\listoftables
\thispagestyle{empty} % pas de numéro de page sur cette page
    
    % annexes
    \newpage
\appendix
\thispagestyle{empty}
\psection{Table des annexes}
\addtocontents{toc}{\protect\setcounter{tocdepth}{0}} % Désactivation de la table des matières

% Personnalisation de la table des annexes
\renewcommand{\stctitle}{}                          % Titre (issue with previous subsection showing up)
\renewcommand\thesubsection{A\arabic{subsection}}   % Numérotation
\renewcommand{\stcSSfont}{}                         % Police normale, pas en gras
\mtcsetrules{secttoc}{off}                          % Désactivation des lignes en haut et en bas de la table

% Affichage de la table des annexes
\secttoc


\newpage
% Annexe 1
\subsection{Annexe A}
Contenu de l'annexe A.

\newpage
% Annexe 2
\subsection{Annexe B}
Contenu de l'annexe B.
    
    \newpage
\AddToShipoutPicture{} 
\thispagestyle{empty} % pas de numéro de page sur cette page
\begin{tikzpicture}[remember picture, overlay]
    \node[anchor=south west, inner sep=0] at (current page.south west) {
        \includegraphics[width=\paperwidth,height=\paperheight]{template/assets/template_last_page.pdf}
    };
    \node at (12.8, -13) {
        \parbox{\textwidth}{
            \sffamily
            \large
            \textbf{INSA TOULOUSE}\\
            135 avenue de Rangueil\\
            31400 Toulouse
            \vspace{0.6em}\\
            Tel: +33 (0)5 61 55 95 13\\
           \href{https://www.insa-toulouse.fr/}{\textbf{www.insa-toulouse.fr}}
        }
    };

    % Des logos cliquables
    \node at (5.53, -15.15) {\href{https://www.facebook.com/INSAToulouse/}{\includegraphics[width=0.8cm]{template/assets/carre.png}}};
    \node at (6.64, -15.15) {\href{https://www.instagram.com/insatoulouse/}{\includegraphics[width=0.8cm]{template/assets/carre.png}}};
    \node at (7.75, -15.15) {\href{https://www.linkedin.com/school/institut-national-des-sciences-appliqu%C3%A9es-de-toulouse/}{\includegraphics[width=0.8cm]{template/assets/carre.png}}};
    \node at (8.89, -15.15) {\href{https://www.youtube.com/user/insatoulouse}{\includegraphics[width=0.8cm]{template/assets/carre.png}}};
\end{tikzpicture} % ne pas toucher
\end{document}
