\documentclass[a4paper,autre]{../texmf/tex/latex/insa/insa}

\usepackage[T1]{fontenc}
\usepackage[french]{babel}
\usepackage{lipsum}

\auteur{Le club informatique}
\uf{Templates latex}
\enseignant{}
\classe{}
\titre{Exemple d'un autre type de document}
\type{Type différent}


\begin{document}
	\maketitle
	\tableofcontents
	
	\section{Introduction}
		\lipsum[1-10]
	\section{Premières notion}
		\lipsum[1-2]
		\subsection{Explication de la première notion}
			\lipsum[1-5]
		\subsection{Démonstration de la première notion}
			\subsubsection{Petit lemme}
				\lipsum[1-2]
			\subsubsection{Lemme un peu plus grand mais utile}
				\lipsum[1-4]
			\subsubsection{Démonstration complète}
				\lipsum[1-5]
	
	\part{Premier thème}
		\section{Premières notion}
			\lipsum[1-2]
			\subsection{Explication de la première notion}
				\lipsum[1-5]
			\subsection{Démonstration de la première notion}
				\subsubsection{Petit lemme}
					\lipsum[1-2]
				\subsubsection{Lemme un peu plus grand mais utile}
					\lipsum[1-4]
				\subsubsection{Démonstration complète}
					\lipsum[1-5]
		\section{Seconde notion}
			\lipsum[1-2]
			\subsection{Explication de la Seconde notion}
				\lipsum[1-5]
			\subsection{Démonstration de la Seconde notion}
				\subsubsection{Petit lemme}
					\lipsum[1-2]
				\subsubsection{Lemme un peu plus grand mais utile}
					\lipsum[1-4]
				\subsubsection{Démonstration complète}
					\lipsum[1-5]
	\part{Deuxième thème}
		\section{Premières notion}
				\lipsum[1-2]
			\subsection{Explication de la première notion}
				\lipsum[1-5]
			\subsection{Démonstration de la première notion}
				\subsubsection{Petit lemme}
					\lipsum[1-2]
				\subsubsection{Lemme un peu plus grand mais utile}
					\lipsum[1-4]
				\subsubsection{Démonstration complète}
					\lipsum[1-5]
		\section{Seconde notion}
			\lipsum[1-2]
			\subsection{Explication de la Seconde notion}
				\lipsum[1-5]
			\subsection{Démonstration de la Seconde notion}
				\subsubsection{Petit lemme}
					\lipsum[1-2]
				\subsubsection{Lemme un peu plus grand mais utile}
					\lipsum[1-4]
				\subsubsection{Démonstration complète}
					\lipsum[1-5]
	\begin{appendix}
	\section{Compléments}
		\lipsum[8-9]
		\section{D'autres compléments}
			\lipsum[52-56]
		\section{Encore d'autres compléments}
			\lipsum[68-92]
		\section{Plein d'autre compléments}
			\lipsum[105-109]
	\end{appendix}


	\makefourthcover
\end{document}